\documentclass[12pt, a4paper, oneside]{ctexart}
\usepackage{amsmath, amsthm, amssymb, appendix, bm, graphicx, hyperref, mathrsfs,adjustbox}
\usepackage{enumerate}
\title{\textbf{论文标题}}
\author{tanghongyu}
\date{\today}
\linespread{1.5}
\newtheorem{theorem}{定理}[section]
\newtheorem{definition}[theorem]{定义}
\newtheorem{lemma}[theorem]{引理}
\newtheorem{corollary}[theorem]{推论}
\newtheorem{example}[theorem]{例}
\newtheorem{proposition}[theorem]{命题}
\renewcommand{\abstractname}{\Large\textbf{摘要}}

\begin{document}

\maketitle

\setcounter{page}{0}
\maketitle
\thispagestyle{empty}

\begin{abstract}
    这里是摘要.
    \par\textbf{关键词:}这里是关键词; 这里是关键词.
\end{abstract}

\newpage
\pagenumbering{Roman}
\setcounter{page}{1}
\tableofcontents
\newpage
\setcounter{page}{1}
\pagenumbering{arabic}

\section{一级标题}

\subsection{二级标题}

\begin{theorem}

\end{theorem}

\begin{table}[htbp]
    \centering  % 显示位置为中间
    \caption{ }  % 表格标题
    \label{tab2}  % 用于索引表格的标签
    %字母的个数对应列数,|代表分割线
    % l代表左对齐,c代表居中,r代表右对齐
    \begin{adjustbox}{max width=\textwidth}
        \begin{tabular}{|c|c|c|}
            \hline
            $ $ &   & \\
            \hline
                &   & \\
            \hline
                &     \\
            \hline
                &   & \\
            \hline
        \end{tabular}
    \end{adjustbox}
\end{table}



\begin{equation}
    \begin{split}
    \end{split}
\end{equation}



\section{附录标题}
这里是附录.



\begin{thebibliography}{99}
    \bibitem{a}作者. \emph{文献}[M]. 地点:出版社,年份.
    \bibitem{b}作者. \emph{文献}[M]. 地点:出版社,年份.
\end{thebibliography}

\begin{appendices}
    \renewcommand{\thesection}{\Alph{section}}
    \section{附录标题}
    这里是附录.
\end{appendices}

\end{document}